\section{Extensions and Unification Theory}

The establishment of this mathematical system has far-reaching implications that bridge mathematics and theoretical physics.

\subsection{Renormalization and Holographic Duality}
The system can be interpreted through the lens of physics. Mersenne primes function as phase transitions in a renormalization flow, defining and resetting the encoding rules at different scales. This provides a compelling candidate for a concrete realization of the holographic principle, where the 1D structure of the Sierpinski triangle's rows encodes the 2D distribution of primes. The Mersenne primes are the "light sources" that structure this projection.

\subsection{A New Computational Substrate}
This system suggests that computation can be performed through geometric transformations in a fractal space, rather than traditional arithmetic logic. The Kéya engine is a first step towards building a computer based on these principles.

\section{Symbolic Homology: Specific Applications}

\subsection{Mathematical Structures as Operator Dynamics}
"This operator algebra also admits a symbolic homotopy interpretation: divergent expressions trace non-trivial loops, with $\lha$ acting as a folding homotopy, $\reflec$ as orientation reversal, and $\fuse$ as cancellation to the null class. This geometric interpretation underlies our treatment of fields, symmetries, and biological systems as dynamic topologies of meaning."

Specific mathematical objects reveal their operator signatures:
\begin{itemize}
    \item \textbf{Sierpiński gasket}: The literal boundary fractal of binomial combinations under symbolic descent.
    \item \textbf{Pascal symmetry}: Curvature is the measure of deflection from perfect Pascal symmetry.
    \item \textbf{Euler's formula}: $e^{i\pi} = -1$ is a fundamental rotation operator in the complex phase space built from generative symbols.
    \item \textbf{Gödel's incompleteness}: Describes the existence of unreachable attractors within finite axiomatic systems.
\end{itemize}

\subsection{Operator Homology}
% TODO: Describe the mapping between mathematical operators.

\subsection{Thermodynamic and Chemical Homology}
The language of the $\sigma$-calculus finds a direct analog in chemical reaction dynamics. Equilibrium is not a static state, but a balanced flux. This isomorphism is so direct that chemical reactions can be simulated as the resolution of an `EquilibriumOperator`. The classic H$_2$ + I$_2$ $\rightleftharpoons$ 2HI reaction, for example, can be modeled by defining forward ($\fuse$ and $\grow$) and reverse ($\lha$ and $*$) operators, which are then resolved to a stable $0$ attractor, perfectly modeling the real-world equilibrium dynamics.

\begin{center}
\renewcommand{\arraystretch}{1.2}
\begin{tabular}{l l}
\hline
\textbf{Chemical Concept} & \textbf{Symbolic $\sigma$ Equivalent} \\
\hline
Forward Reaction & Growth operators ($\grow$, $\tensor$, $\fuse$) \\
Reverse Reaction & Descent or reflection ($\lha$, $\reflec$) \\
Reaction Rate & Curvature gradient of the operator field \\
Equilibrium Constant $K_{eq}$ & A fixed-point attractor in $\sigma$-space \\
Dynamic Equilibrium & Self-cancelling symbolic fusion ($a \fuse \reflec a \to \void$) \\
Catalyst & Operator compression via attractors ($\gold$, $\pirot$) \\
\hline
\end{tabular}
\end{center}

\subsection{Quantum Homology}
The principle of self-cancelling equilibrium resonates deeply with concepts in quantum field theory, suggesting that kéya describes a fundamental condition of stability across physical systems.
\begin{itemize}
    \item \textbf{Feynman Path Interference:} The cancellation of paths via destructive interference is equivalent to the fusion of an operator with its reflection ($\op{A} \fuse \reflec\op{A}$).
    \item \textbf{Vacuum Fluctuation:} The spontaneous creation and annihilation of particle-antiparticle pairs is a physical manifestation of $\grow \fuse \lha \to \void$.
    \item \textbf{Gauge Symmetry:} The invariance of a system under a transformation is modeled by the fusion of that transformation with its inverse, resolving to the identity ($\op{T} \fuse \op{T}^{-1} \to \sym{1}$).
\end{itemize}

\subsection{Biological Homology}
The principles of kéya lead to directly testable hypotheses across virology and genetics. By translating biological systems into their $\sigma$-signatures, we can predict their behavior as a form of symbolic field dynamics.

\begin{center}
\renewcommand{\arraystretch}{1.2}
\begin{tabular}{l l l}
\hline
\textbf{System} & \textbf{$\sigma$ Signature} & \textbf{Testable Hypothesis} \\
\hline
Crucivirus X-genome & $\tensor(\text{DNA, RNA}) \fuse \lha(\rotate)$ & Rolling-circle replication is a $\pi$-phase loop. \\
Retrovirus integration & $\reflec(\text{Host}) \fuse \repflow(\text{RNA}) \fuse \lha(\text{DNA})$ & Integration seeks reflective attractor equilibrium. \\
Human Endogenous & $\void \fuse \lha(\text{HostRNA})$ & Dormant viral elements act as \\
Retroviruses (HERVs) & & curvature locks for genome flow. \\
Ribosome translation & $\lha(\repflow(\text{mRNA})) \fuse \text{tRNA}$ & Protein folding is literal symbolic descent. \\
\hline
\end{tabular}
\end{center}

\subsection{Computational Homology}
% TODO: Discuss the relationship between $\sigma$-calculus and formal computation models. 
