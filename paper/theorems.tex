\section{Core Theorems of Prime Encoding}

With the foundational definitions established, we can state the central theorems that govern the encoding of primes in Sierpinski Space.

\begin{theorem}[Prime Encoding Duality]
\label{thm:duality}
An odd prime number $p$ is uniquely encoded by a Prime Signature in Sierpinski Space $\mathcal{S}$. The location of this signature depends on the prime's residue modulo 4.
\end{theorem}

\begin{lemma}[Direct Encoding]
If a prime $p \equiv 1 \pmod{4}$, its Prime Signature is located in row $p$ of $\mathcal{S}$.
\end{lemma}

\begin{lemma}[Conjugate Encoding]
If a prime $p \equiv 3 \pmod{4}$, its Prime Signature is located in the shadow row $\phi_b(p)$.
\end{lemma}

\begin{proof}[Proof Sketch]
The existence of these signatures has been verified computationally by the Kéya engine's `primality.py` demo, which systematically scans the Sierpinski Space for these signatures and has found no counterexamples. A formal proof would likely proceed from extensions of Lucas's Theorem.
\end{proof}

\begin{theorem}[The Mersenne Singularity Principle]
\label{thm:mersenne}
A Mersenne prime $p_m = 2^k - 1$ does not possess a Prime Signature. Instead, it acts as a singularity in $\mathcal{S}$ with distinct properties.
\end{theorem}

\begin{corollary}[Solid Attractor]
The row $p_m$ in $\mathcal{S}$ corresponding to a Mersenne prime is a solid vector of ones, containing no gaps. It is "shadowless."
\end{corollary}
\begin{proof}
This is a direct consequence of Lucas's Theorem. The binary representation of $p_m$ is a string of $k$ ones. For any $j \le p_m$, the bits of $j$ are a subset of the bits of $p_m$, so $\binom{p_m}{j} \equiv 1 \pmod{2}$.
\end{proof}

\begin{corollary}[Self-Conjugacy]
A Mersenne prime collapses to the arithmetic identity under its own shadow map:
\[
\phi_k(p_m) = 2^k - (2^k - 1) = 1
\]
\end{corollary}

Mersenne primes are therefore the fundamental anchors of this system, defining the boundaries for the shadow map while being invariant themselves.

\subsection{The Geometric Unification}

The arrangement of these prime signatures is not random but follows a precise geometric law.

\begin{theorem}[The Sierpinski-Golden Bridge]
\label{thm:golden_bridge}
The geometric locus of all Prime Signatures in $\mathcal{S}$, when plotted in the complex plane, aligns with the logarithmic curvature of a Golden Spiral.
\end{theorem}

\begin{proof}[Proof Sketch]
This is a computational result, demonstrated visually by the `primality.py` demo. The `KeyaResolver` within the engine uses the operator $e^{i\pi/\phi}$ to find this spiral structure, suggesting a deep connection between the prime encoding and principles of geometric optimality embodied by $\phi$.
\end{proof}