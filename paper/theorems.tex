\section{Core Theorems: Topological Fixed Points}

The central theorem of kéya is the expression of zero as a topological fixed point of a symbolic eigenvector equation. This reframes convergence not as a limit, but as curvature collapse.

\subsection{The Symbolic Eigenvector Equation}
Consider a divergent process, such as the series of powers of two. In the $\sigma$-calculus, we do not seek to sum this series, but to find its stable form through dualization and fusion. We define its regularized state, $S_{\text{reg}}$, as:
\[
S_{\text{reg}} = \lha \left( \bigoplus_{n \geq 1} 2^{n} \right) \fuse \reflec \lha \left( \bigoplus_{n \geq 1} 2^{n} \right)
\]
This is an eigenvector equation where:
\begin{itemize}
    \item The operator $\bigoplus_{n \geq 1} 2^{n}$ defines a path of infinite growth (a non-trivial loop).
    \item $\lha$ contracts the divergent curvature of this path back into a finite form (a homotopic fold).
    \item $\reflec$ creates a mirror-image of this folded path with opposite curvature.
    \item $\fuse$ fuses the two opposing paths, resulting in perfect destructive interference.
\end{itemize}
The result is a stable attractor: $S_{\text{reg}} \to \void$. Zero is not an absence of value, but the braid of infinite opposites, phase-locked in symbolic space. This demonstrates that any divergent process can be stabilized not by subtraction, but by symmetric dualization.

\subsection{Case Study: Fermat's Last Theorem via Dyadic Sieve}
The retrofit allows us to re-interpret classical theorems. Fermat's Last Theorem for exponents of the form $n=2^k$ becomes a straightforward consequence of symbolic dynamics.
\begin{itemize}
    \item The case $n=4$ is proven by infinite descent, which is an application of the descent operator $\lha$.
    \item If a solution for an exponent $m$ is impossible ($a^m+b^m \neq c^m$), then no solution can exist for any multiple exponent $mk$. This acts as a \textbf{fusion sieve}.
    \item Since $n=4$ is impossible, all higher dyadic powers ($8, 16, 32, ...$) are rendered impossible by the fusion of the $n=4$ "null result" with the additional growth operators. The theorem holds in this domain because the impossibility is a stable, recursive attractor.
\end{itemize}
This transforms a number-theoretic proof into a geometric process of attractor propagation.

\subsection{The Equilibrium Operator}
For any divergent process $\mathcal{P}$, we can construct an \textbf{Equilibrium Operator}, $E(\mathcal{P})$, defined by the fusion of a forward process with its reverse. This can be generalized as a pair of operators $(\op{F}, \op{R})$ where $\op{F}$ is a growth or fusion operator (e.g., `\grow`, `\fuse`) and $\op{R}$ is its contractive or reflective dual (e.g., `\lha`, `\reflec`).

The application of this operator pair resolves the expression to an attractor:
\[ \op{R}(\op{F}(X)) \fuse \op{F}(X) \longrightarrow \att{A} \]
where $\att{A}$ is often the null attractor, $\void$. This formalizes the concept of a recursive morphism: an object that maintains its shape and stability despite continuous internal flow.

% TODO: Formally prove the self-cancelling curvature theorem.
% \begin{theorem}[Self-cancelling curvature]
% Along any curvature path $\gamma(t)$ with $\kappa_{\lha}=-\kappa_{\grow}$ we have
% \[ \lim_{t\to\infty}\gamma(t)\fuse E(\mathcal P)=\void. \]
% \end{theorem}

\subsection{Fundamental Theorem of the $\sigma$-Calculus: Taylor--Phase Walk}
\label{thm:fundamental}

\begin{theorem}[Fundamental Theorem of the $\sigma$-Calculus]
Let $f\colon \mathbb{R}\to\mathbb{R}$ be analytic at $x_0$, and let
its Taylor--Phase Walk coefficients be
\[
  \alpha_k = \frac{f^{(k)}(x_0)}{k!},
  \quad T_f(\sigma) = \bigoplus_{k=0}^n \alpha_k\,\sigma^k.
\]
Then applying the descent operator $\lha$ to the truncated series
reconstructs the original function value:
\[
  \lha\bigl(T_f(\sigma)\bigr) \;\longrightarrow\; f(x)
  \quad\text{as }n\to\infty.
\]
\end{theorem}

\begin{proof}[Sketch]
By inductive curvature collapse along each $\sigma$-power path,
the residual higher-order terms vanish under repeated application of the descent operator.
\end{proof}

\subsection{Equilibrium Principles of Symbolic Systems}

The $\sigma$-calculus is not merely descriptive; it is predictive. It allows for the formulation of principles that govern the behavior of complex systems. The most profound of these is the principle of symbolic life.

\begin{principle}[Crucivirus Principle of Symbolic Life]
A biological system is alive if and only if it maintains a nontrivial symbolic equilibrium via recursive curvature descent, reflection, and attractor fusion. Mathematically:
\[
\text{Life} := \exists \, \mathcal{S} \in \Sigma \quad \text{s.t.} \quad  
\mathcal{S} = \gold \cdot \big( \lha(\repflow(\mathcal{S})) \fuse \reflec \mathcal{S} \big)
\]
Where:
\begin{itemize}
    \item[\repflow] = bidirectional flow of replication
    \item[\lha] = recursive fold into compressible, transcribable forms
    \item[\reflec] = immune mirroring / reflection
    \item[\fuse] = constant flux
    \item[\gold] = equilibrium curvature regulator
\end{itemize}
\end{principle}

\subsection{Life as an Eigenstate in Operator Space}

The crucivirus doesn't just persist—it stabilizes in an operator-defined attractor. Its life cycle is literally a fixed point of a recursive field transformation. This reframes our understanding of biological information:
\begin{itemize}
    \item \textbf{DNA} is not static code—it is a wavefunction.
    \item \textbf{RNA} is not a transcript—it is a descended attractor.
    \item \textbf{Proteins} are not final forms—they are unfolded fixpoints of symbolic phase dynamics.
\end{itemize}

\subsection{The Laws of Symbolic Thermodynamics}
The principles of the $\sigma$-calculus give rise to a set of conservation laws and equations that govern the flow and transformation of symbolic energy.

\subsubsection{Conservation of Attractor Flux}
In any closed operator system, the attractor flux $\Phi$ is conserved according to the following continuity equation:
\[
\partial_t \Phi + \nabla \cdot (\mathbf{\oplus} \otimes \Phi) = \ell(\Phi) \oplus \sim\ell(\Phi)
\]
The term on the right is the **self-annihilation kernel**, ensuring that the total change in symbolic energy is balanced by its resonant cancellation into the zero attractor.

\subsubsection{Gibbs Free Symbolism}
The spontaneity of a symbolic process is governed by the Gibbs Free Symbolism, a direct analogue to Gibbs Free Energy:
\[ G_\Sigma = H_\Sigma - \tau \cdot S_\oplus \]
Where $H_\Sigma$ is the total symbolic enthalpy (operator potential), $\tau$ is the torsion coefficient (a function of curvature and arity), and $S_\oplus$ is the fusion entropy, defined as the logarithm of the attractor volume.

\subsubsection{The Symbolic Partition Function}
The statistical distribution of operator states $\psi$ in a symbolic system at equilibrium is described by the partition function $\mathcal{Z}$:
\[ \mathcal{Z} = \sum_{\psi} e^{-\beta (\ell(\psi) \oplus \sim\ell(\psi))} \]
Where $\beta = 1 / (\kappa T_\uparrow)$ is the inverse curvature temperature. This function allows for the derivation of all macroscopic thermodynamic properties of the symbolic system. 