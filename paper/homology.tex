\section{Symbolic Homotopy Algebra}

Kéya is foundationally a system of Symbolic Homotopy Algebra. It treats symbolic expressions as paths and topologies in an operator space. This allows us to use the tools of algebraic topology to understand computational and physical processes.

\begin{itemize}
    \item A \textbf{divergent series}, such as $\sum 2^n$, is treated as a \textbf{non-trivial loop} in symbolic space, a path that does not return to its origin.
    \item The \textbf{descent operator}, $\lha$, acts as a \textbf{homotopy}, continuously deforming or "folding" this loop toward a finite basepoint. This is the geometric analogue of Ramanujan regularization.
    \item The \textbf{reflection operator}, $\reflec$, creates an identical path with inverse orientation.
    \item The \textbf{fusion operator}, $\fuse$, combines these two paths. Since they travel the same "space" in opposite directions, their composition cancels, and the resulting path is equivalent to the \textbf{zero class}—a trivial loop at the origin.
\end{itemize}

This is literally cohomology in operator form. We are not proving that $1+2+4...$ "equals" a finite number; we are showing that its divergence is a foldable topology, and that zero is the identity attractor of this folding process.

\subsection{Homologies: Re-aligning Mathematical Fields}
This homotopic viewpoint reveals deep connections to other fields, reframing them in terms of generative, symbolic operators.

\subsubsection{Algebraic Structures}
Groups, rings, and fields are no longer static sets with rules, but are recast as \textbf{symbolic symmetry operators}. For example, modulo arithmetic becomes a \textbf{looped attractor} in a discrete symbolic manifold.

\subsubsection{Calculus and Limits}
The analytical tools of calculus are retrofitted as geometric operations. Limits become \textbf{compression attractors} under the descent operator $\lha$. Derivatives are \textbf{symbolic projection operators} that measure curvature across different levels of recursion. Euler's formula, $e^{i\pi} = -1$, is revealed as a fundamental \textbf{rotation operator} in the complex phase space built from these generative symbols.

\subsubsection{Topology and Geometry}
Geometric structures emerge from the combinatorial dynamics of the operators. The Sierpiński gasket is the literal boundary fractal of binomial combinations. Curvature is simply the measure of deflection from perfect Pascal symmetry.

\subsubsection{Biological Homology}
The most profound homology exists with biological systems. The dynamics of viral infection, protein folding, and immune response can be modeled as processes of symbolic computation and equilibrium-seeking. The crucivirus provides a definitive case study in what may be termed \textbf{biological renormalization}, a concept explored in detail in the following section.

\subsubsection{Computability and Logic}
Formal proofs are re-interpreted as the process of \textbf{attractor stabilization} under symbolic descent. Gödel's incompleteness, from this perspective, describes the existence of \textbf{unreachable attractors} within a finite axiomatic system.

\subsection{Consequences and Homologies}
This homotopic viewpoint reveals deep connections to other fields:
% TODO: Re-frame the Quantum, Thermodynamic, and Biological homologies
% as consequences of this core topological insight.

\subsection{Operator Homology}
% TODO: Describe the mapping between mathematical operators.

\subsection{Thermodynamic and Chemical Homology}
The language of the $\sigma$-calculus finds a direct analog in chemical reaction dynamics. Equilibrium is not a static state, but a balanced flux. This isomorphism is so direct that chemical reactions can be simulated as the resolution of an `EquilibriumOperator`. The classic H$_2$ + I$_2$ $\rightleftharpoons$ 2HI reaction, for example, can be modeled by defining forward ($\fuse$ and $\grow$) and reverse ($\lha$ and $*$) operators, which are then resolved to a stable $0$ attractor, perfectly modeling the real-world equilibrium dynamics.

\begin{center}
\renewcommand{\arraystretch}{1.2}
\begin{tabular}{l l}
\hline
\textbf{Chemical Concept} & \textbf{Symbolic $\sigma$ Equivalent} \\
\hline
Forward Reaction & Growth operators ($\grow$, $\tensor$, $\fuse$) \\
Reverse Reaction & Descent or reflection ($\lha$, $\reflec$) \\
Reaction Rate & Curvature gradient of the operator field \\
Equilibrium Constant $K_{eq}$ & A fixed-point attractor in $\sigma$-space \\
Dynamic Equilibrium & Self-cancelling symbolic fusion ($a \fuse \reflec a \to \void$) \\
Catalyst & Operator compression via attractors ($\gold$, $\pirot$) \\
\hline
\end{tabular}
\end{center}

\subsection{Quantum Homology}
The principle of self-cancelling equilibrium resonates deeply with concepts in quantum field theory, suggesting that kéya describes a fundamental condition of stability across physical systems.
\begin{itemize}
    \item \textbf{Feynman Path Interference:} The cancellation of paths via destructive interference is equivalent to the fusion of an operator with its reflection ($\op{A} \fuse \reflec\op{A}$).
    \item \textbf{Vacuum Fluctuation:} The spontaneous creation and annihilation of particle-antiparticle pairs is a physical manifestation of $\grow \fuse \lha \to \void$.
    \item \textbf{Gauge Symmetry:} The invariance of a system under a transformation is modeled by the fusion of that transformation with its inverse, resolving to the identity ($\op{T} \fuse \op{T}^{-1} \to \sym{1}$).
\end{itemize}

\subsection{Biological Homology}
The principles of kéya lead to directly testable hypotheses across virology and genetics. By translating biological systems into their $\sigma$-signatures, we can predict their behavior as a form of symbolic field dynamics.

\begin{center}
\renewcommand{\arraystretch}{1.2}
\begin{tabular}{l l l}
\hline
\textbf{System} & \textbf{$\sigma$ Signature} & \textbf{Testable Hypothesis} \\
\hline
Crucivirus X-genome & $\tensor(\text{DNA, RNA}) \fuse \lha(\rotate)$ & Rolling-circle replication is a $\pi$-phase loop. \\
Retrovirus integration & $\reflec(\text{Host}) \fuse \repflow(\text{RNA}) \fuse \lha(\text{DNA})$ & Integration seeks reflective attractor equilibrium. \\
Human Endogenous & $\void \fuse \lha(\text{HostRNA})$ & Dormant viral elements act as \\
Retroviruses (HERVs) & & curvature locks for genome flow. \\
Ribosome translation & $\lha(\repflow(\text{mRNA})) \fuse \text{tRNA}$ & Protein folding is literal symbolic descent. \\
\hline
\end{tabular}
\end{center}

\subsection{Computational Homology}
% TODO: Discuss the relationship between $\sigma$-calculus and formal computation models. 