\section{Core Theorems: Attractor Field Theory}

The central theorem of kéya is the expression of zero as a topological fixed point of a symbolic eigenvector equation. This reframes convergence not as a limit, but as curvature collapse within a **Banach algebra of curvature operators**.

\subsection{Symbolic Eigenvectors \& Topological Fixed Points}
We define the symbolic vector space as a complete normed algebra:
\[
\mathcal{V} = \left\{ \bigoplus_k \alpha_k \sigma^k  \mid  \| \alpha \|_{\ell^2} < \infty \right\}, \quad \lha : \mathcal{V} \to \mathcal{V}
\]

Consider a divergent process, such as the series of powers of two. In the $\sigma$-calculus, we do not seek to sum this series, but to find its stable form through dualization and fusion. We define its regularized state, $S_{\text{reg}}$, as:
\[
S_{\text{reg}} = \lha \left( \bigoplus_{n \geq 1} 2^{n} \right) \fuse \reflec \lha \left( \bigoplus_{n \geq 1} 2^{n} \right)
\]

\begin{theorem}[Universal Stabilization]
For any divergent process $\mathcal{P} = \bigoplus_k \alpha_k \in \mathcal{V}$,
\[ \lha(\mathcal{P}) \fuse \reflec \lha(\mathcal{P}) \xrightarrow{\text{norm}} \void \]
is the unique fixed point under the $\sigma$-operator algebra $\{\lha, \fuse, \reflec\}$.
\end{theorem}

This demonstrates that any divergent process can be stabilized not by subtraction, but by symmetric dualization through curvature operators. The zero attractor $\void$ emerges as the **unique fixed point** under the composition $\lha \circ \reflec \circ \fuse$.

\subsection{Fermat Sieve as Attractor Inheritance}
The retrofit allows us to re-interpret classical theorems through **attractor inheritance**. Fermat's Last Theorem for exponents of the form $n=2^k$ becomes a consequence of symbolic dynamics.

\begin{theorem}[Dyadic Impossibility Propagation]
If $n=4$ is an impossibility attractor in the exponent semigroup $\langle 2 \rangle$, then for any $m = 2^k$ where $k \geq 2$:
\[ \text{Impossible}(4) \fuse \grow^{\otimes k} \longrightarrow \text{Impossible}(m) \]
The impossibility is a stable, recursive attractor that propagates through the fusion sieve.
\end{theorem}

This transforms number-theoretic impossibility into geometric attractor inheritance, where the descent operator $\lha$ acts on the exponent semigroup to preserve the impossibility structure.

\subsection{$\sigma$-Calculus Reconstruction Theorem}
The fundamental reconstruction theorem employs **curvature-aware Taylor reconstruction**, where $\lha$ compresses high-frequency symbolic noise through resolvent formalism.

\begin{theorem}[Curvature-Completion]
Let $f\colon \mathbb{R}\to\mathbb{R}$ be analytic at $x_0$, with Taylor--Phase Walk coefficients
\[
  \alpha_k = \frac{f^{(k)}(x_0)}{k!}, \quad T_f(\sigma) = \bigoplus_{k=0}^n \alpha_k\,\sigma^k.
\]
Then the descent operator reconstructs $f$ via resolvent integration:
\[
  \lha\bigl(T_f(\sigma)\bigr) = f(x) + O(\sigma^{n+1})
\]
where $\lha = \oint_C \frac{R(\zeta, T_f)}{\zeta} d\zeta$ and $C$ encircles the function's attractor basin.
\end{theorem}

\begin{proof}[Sketch]
By resolvent formalism, the descent operator compresses residual higher-order terms through curvature collapse. The contour integral over the attractor basin ensures convergence in the symbolic Sobolev norm.
\end{proof}

\subsection{Automorphic Equilibrium Theorem}
We reframe biological dynamics as **automorphic equilibrium** to avoid bioconflation while capturing self-organizing dynamics.

\begin{theorem}[Automorphic Equilibrium]
A system $\mathcal{S}$ admits automorphic equilibrium if and only if:
\[ \mathcal{S} \approx_{\epsilon} \gold \cdot \left( \lha(\repflow(\mathcal{S})) \fuse \reflec \mathcal{S} \right) \]
where $\approx_{\epsilon}$ denotes curvature $\epsilon$-equivalence, and:
\begin{itemize}
   \item[$\repflow$] = curvature diffusion (entropy operator)
   \item[$\lha$] = recursive fold into compressible forms
   \item[$\reflec$] = phase mirroring operator
   \item[$\fuse$] = constant flux equilibrium
   \item[$\gold$] = torsion compensator (stabilizes phase drift)
\end{itemize}
\end{theorem}

\subsection{Noether's Theorem for Symbolic Systems}
Every continuous symmetry of the operator algebra corresponds to a conserved curvature current.

\begin{theorem}[Symbolic Noether Conservation]
For every continuous symmetry $\delta \mathcal{S} = \epsilon \cdot \mathcal{Q}(\mathcal{S})$ of the operator algebra $\{\lha, \fuse, \reflec\}$, there exists a conserved curvature current:
\[ J_\mu = \frac{\delta \mathcal{L}}{\delta (\partial_\mu \mathcal{S})} \mathcal{Q}(\mathcal{S}) \]
where $\partial_\mu J_\mu = 0$.
\end{theorem}

\textbf{Example (Scale Invariance):} Scale symmetry $\sigma \mapsto \lambda \sigma$ generates conservation of symbolic complexity: $\partial_\mu J_\mu^{\text{scale}} = 0$.

\subsection{Symbolic-Holographic Duality}
The attractor basin exhibits holographic correspondence with boundary operator algebras.

\begin{theorem}[Holographic Attractor Duality]
The automorphism group of the symbolic vector space $\mathcal{V}$ is dual to the residue algebra on its boundary:
\[ \text{Aut}(\mathcal{V}) \cong \text{Res}(\partial \mathcal{V}) \]
where $\partial \mathcal{V}$ is the symbolic boundary defined by attractor basin topology.
\end{theorem}

\textbf{Corollary (Fermat Boundary Anomaly):} Fermat's impossibility manifests as a boundary anomaly in the exponent algebra, where the holographic dual encodes the impossibility structure.

\subsection{Fractal Renormalization}
Repeated application of $\lha$ induces renormalization group flow in curvature space.

\begin{theorem}[Symbolic Renormalization Flow]
The descent operator $\lha$ generates a renormalization group flow in the curvature coupling space:
\[ \beta(g) = \mu \frac{d g}{d \mu} = - \epsilon g + C g^2 \]
where $g$ is the curvature coupling constant. Fixed points correspond to universal attractor classes.
\end{theorem}

\textbf{Corollary (Dyadic Universality):} The dyadic impossibility class $\{4, 8, 16, 32, \ldots\}$ forms a universal attractor under the renormalization flow.

\subsection{The Laws of Symbolic Thermodynamics}
The principles of the $\sigma$-calculus give rise to unified conservation laws governing symbolic energy transformation.

\subsubsection{First Law: Conservation of Curvature Flux}
\begin{theorem}[Attractor Flux Conservation]
In any closed operator system, the curvature flux satisfies:
\[
\partial_t \Phi + \nabla \cdot \mathbf{J}_{\oplus} = \ell(\Phi) \oplus \sim\ell(\Phi)
\]
where $\mathbf{J}_{\oplus} = \Phi \otimes \text{Res}(\nabla \sigma)$ is the fusion current, and the right side represents self-annihilation balance.
\end{theorem}

\subsubsection{Gibbs Free Symbolism}
The spontaneity of symbolic processes follows from the Gibbs Free Symbolism:
\[ G_\Sigma = H_\Sigma - \tau \cdot S_\oplus \]
where $H_\Sigma$ is symbolic enthalpy (operator potential), $\tau$ is the torsion coefficient (curvature-arity function), and $S_\oplus$ is fusion entropy (logarithm of attractor volume).

\subsubsection{The Symbolic Partition Function}
Statistical equilibrium is governed by the path integral over attractor basins:
\[ \mathcal{Z} = \sum_{\psi} e^{-\beta (\ell(\psi) \oplus \sim\ell(\psi))} \]
where $\beta = 1 / (\kappa T_\uparrow)$ is the inverse curvature temperature, allowing derivation of all macroscopic thermodynamic properties.

\subsection{Integration with Ricci Reinvestment Theory}
The symbolic thermodynamics integrates seamlessly with geometric field theory:
\[
\underbrace{\nabla_\mu T^{\mu\nu} = 0}_{\text{classical}} \longrightarrow \underbrace{\nabla_\mu \left( T^{\mu\nu} \oplus \Psi(\mathbb{A}) \right) = 0}_{\text{Kéya-compatible}}
\]
where $\Psi(\mathbb{A})$ is the attractor reinvestment operator encoding curvature memory, ensuring energy conservation through $\fuse$-balance of stress-energy and symbolic dynamics. 